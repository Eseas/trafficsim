\documentclass{beamer}

\setbeamerfont{framesubtitle}{size=\Large}
\setbeamerfont{frametitle}{size=\huge}


\usepackage{tikz}
\usepackage{verbatim}

\RequirePackage{polyglossia} 
\setdefaultlanguage{english}
\setotherlanguage{lithuanian}

\title{An Assessment Method for Human-Like Behavior of Non-Player Characters in Vehicle Traffic Simulators}
\subtitle{Eismo simuliacijose simuliuojamų eismo dalyvių žmogiškos elgsenos vertinimo metodas}
\author[shortname]{Kazimieras Senvaitis \and \\ Vadovas: Asist., Dr. Vytautas Valaitis  \and \\ Recenzentas: Partn. Prof., Dr. Vytautas Ašeris}
\date{2018-12-19}

\addtobeamertemplate{frametitle}{}{
	\begin{tikzpicture}[remember picture,overlay]
	\node[anchor=north east,yshift=2pt] at (current page.north east) {\includegraphics[height=1.5cm]{mif-logo.png}};
	\end{tikzpicture}}

\begin{document}

	
\maketitle


\section{Dalykinės srities problematika}

\subsection{Autonominių automobilių mokymas}
\begin{frame}{\insertsection}
\framesubtitle{\insertsubsection}
\begin{itemize}
	\item Autonominio vairavimo tyrimai fiziniame pasaulyje suvaržyti infrastruktūros kainos bei logistikos sunkumų;
	\item Vienu automobiliu surinktos informacijos kiekis yra nepakankamas padengti įvairiems kritiniams atvejams, kurie turi būti apdoroti mokant ir validuojant dirbtinį intelektą;
	\item Daugumai tyrėjų autonominų automobilių mokymas ir validavimas fizinėje aplinkoje yra nepasiekiamas.
\end{itemize}
\end{frame}



\section{Esami sprendimai}

\subsection{Autonominių automobilių mokymas}
\begin{frame}{\insertsection}
\framesubtitle{\insertsubsection}
\begin{itemize}
	\item Modernūs AAA tipo atviro pasaulio kompiuteriniai žaidimai;
	\item Mikroskopinių 2D eismo simuliatorių integravimas su 3D aplinka;
	\item Specialūs simuliatoriai autonominių automobilių mokymui.
\end{itemize}
\end{frame}



\section{Esamų sprendimų problematika}

\subsection{Autonominių automobilių mokymas}
\begin{frame}{\insertsection}
\framesubtitle{\insertsubsection}
\begin{itemize}
	\item Simuliatoriaus realistiškumas priklauso nuo simuliuojamų eismo dalyvių elgsenos realistiškumo;
	\item Simuliuojamų eismo dalyvių elgsena yra paremta griežtomis mažai varijuojančiomis taisyklėmis, sudarytomis iš tikrovės neatspindinčių prielaidų;
	\item Sunku tobulinti eismo dalyvių elgsenos realistiškumą neturit aspektų bei metrikų;
	\item Autonominių automobilių intelektas apmokytas naudojant esamus sprendimus tikėtinai yra per daug ydingas naudojimui realiame eisme.
\end{itemize}
\end{frame}



\section{Tyrimo objektas}


%\subsection{Autonominių automobilių mokymas}
\begin{frame}{\insertsection}
\framesubtitle{\insertsubsection}
Eismo simuliatoriuose, skirtuose autonominių automobilių mokymui, simuliuojamų eismo dalyvių žmogiška elgsena, įskaitant:
\begin{itemize}
	\item Žmogiško vairavimo aspektus;
	\item Žmogiško vairavimo matavimą;
	\item Simuliuojamų eismo dalyvių algoritmus;
	\item Žmogiškos elgsenos metrikų skaitymas iš eismo simuliacijų.
\end{itemize}
\end{frame}


\section{Tikslas ir uždaviniai}
\begingroup
\small% \small in 11pt base font is 10pt
\begin{frame}{\insertsection}
	\framesubtitle{\insertsubsection}
	\vspace{-30.5pt}
%	\textbf{Tikslas}\\
	
	Sukurti eismo simuliacijose simuliuojamų eismo dalyvių žmogiškos elgsenos vertinimo metodą. Iškelti uždaviniai:
	
%	\textbf{Užduotys}
	\begin{enumerate}
		\item Naudojant Goal-Oriented Requirements Language (GRL) dekomponuoti vairavimo užduotis simuliuojamų eismo dalyvių algoritmuose, simuliatoriuose CARLA ir SUMO;
		\item Sukurti įsitikinimo-noro-intencijos (angl. belief-desire-intention (BDI)) modelį apibūdinantį žmogiško vairavimo savybes;
		\item Sukurti BDI modeliu paremtus scenarijus specifinių vairavimo užduočių analizei;
		\item Paruošti vairavimo aplinką leidžiančią žmonėms atlikti specifines vairavimo užduotis ir leidžiančią matuoti jų įvykdymą;
		\item Atlikti žmogaus vairavimo charakteristikų surinkimą naudojant paruoštą vairavimo aplinką;
		\item Sintezuoti žmogaus vairavimo charakteristikas į vertinimo metodą;
		\item Naudojant vertinimo metodą įvertinti CARLA simuliatorių ir SUMO simuliatorių integruotą su Unity3D.
	\end{enumerate}
\end{frame}
\endgroup

\section{Laukiami rezultatai}
\begingroup
\small% \small in 11pt base font is 10pt
\begin{frame}{\insertsection}
\framesubtitle{\insertsubsection}
\vspace{-30.5pt}
\begin{enumerate}
	\item Dekomponuotos vairavimo užduotys esančios simuliuojamų eismo dalyvių algoritmuose, CARLA ir SUMO simuliatoriuose;
	\item Sukurtas belief-desire-intention (BDI) modelis apibūdinantis žmogiškumo savybes vairavime, išryškinantis aspektus, kurių trūksta eismo simuliatoriams;
	\item Sukurti scenarijai orientuoti į žmogiškos elgsenos pasireiškimą ir leidžiantys analizuoti ir matuoti žmogaus vairavimo charakteristikas;
	\item Paruošta vairavimo aplinka su vykdymo metrikų įrašymu, leidžianti žmonėms atlikti specifinius scenarijus;
	\item Atliktas eksperimentas žmogaus vairavimo savybėms surinkti, surinkti matavimai;
	\item Sintezuotos žmogaus vairavimo savybės ir suformuluotas vertinimo metodas žmogiškumo vertinimui specifiniuose scenarijuose;
	\item Naudojant suformuluotą žmogiškumo savybių vertinimo metodą palyginti simuliuojami eismo dalyviai simuliatoriuose CARLA ir SUMO integruotame su Unity3D, pateiktos įžvalgos tobulinimui.
\end{enumerate}
\end{frame}
\endgroup

%\section{Results and conclusions}

%\begin{frame}
%	\frametitle{Results}
%	\begin{enumerate}
%		\item item 1
%	\end{enumerate}
%\end{frame}
%
%\begin{frame}
%	\frametitle{Conclusions}
%	\begin{enumerate}
%		\item item 2
%	\end{enumerate}
%\end{frame}


\end{document}